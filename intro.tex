\section{Вступление}

Задача выявления заметных объектов на изображении (Salient object detection, SOD problem) направлена на обнаружение и выделение частей изображения, наиболее привлекающих внимание
человека \cite{Is-Depth-Really-Necessary-for-SOD}. Можно сказать, что решая её, мы пытаемся имитировать визуальное восприятие человека для определения одного 
или нескольких объектов, привлекающих наибольшее внимаение. Она играет ключевую роль в широком спектре задач компьютреного зрения, таких как:
stereo matching [11], image understanding [12], co-saliency detection [13], action recognition [14], video detection and
segmentation [15]–[18], semantic segmentation [19], [20],
medical image segmentation [21]–[23], object tracking [24],
[25], person re-identification [26], [27], camouflaged object
detection [28], image retrieval [29], etc. [https://arxiv.org/pdf/2008.00230.pdf - переписать и вставить ссылки]

Многие существующие [[19]–[35] из https://arxiv.org/pdf/1907.06781v2.pdf] методы решения задачи SOD используют только информацию, доступную на RGB изображении.
Но кроме изображения кажется естественным использовать информацию с карты глубины(depth map) - изображения, которое хранит расстояние от запечетлённых объектов до камеры.
Карты глубины получают либо с помощью специальных камер глубины, таких как ToF-камеры (Time of Flight) или структурированные световые камеры [TODO ссылки], либо с помощью
стереопары изображений. Имея карту глубины можно получить много дополнительной информации, которая будет полезна для решения задачи SOD, ведь естественно предположить, что объекты на переднем плане должны быть более заметны, 
чем те, что находятся дальше от наблюдателя. [ Более того, с помощью карты глубины можно преодолеть challenging factors, such as complicated
background or different lighting conditions in the scenes. ]

Задача SOD, для решения которой кроме RGB изображения используют карту глубины, называют RDB-D saliency detection problem.
Из-за распространениея устройств, способных строить карты глубины - в основном современных смартфонов - задача выявления заметных объектов на изображении на основе RGB-изображения и карты глубины (RGBD-SOD) привлекает всё большее внимание исследователей. [TODO]

[TODO: кратко описать общие подходы и рассказать про проблемы?]

В данной работе мы рассмотрим существующие подходы для решения задачи RGB-D SOD ... и попробуем улучшить одну из State-of-the-art моделей.

Структура работы следующая:  ...