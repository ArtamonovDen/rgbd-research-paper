\section{Вступление}

Задача выявления заметных объектов на изображении (Salient Object Detection, SOD problem) направлена на
 обнаружение и выделение частей изображения, наиболее привлекающих внимание
человека \cite{Is-Depth-Really-Necessary-for-SOD}. Можно сказать, что, решая её, мы пытаемся имитировать визуальное восприятие человека для определения одного 
или нескольких объектов, привлекающих наибольшее внимание. Задача SOD играет ключевую роль в широком спектре задач компьютреного зрения, таких как:
распознавание изображений\cite{Image-Understanding}, стереоконструкция\cite{Stereo-Matching},
распознавание действий\cite{Action-Recognition}, семантическая сегментация \cite{Semantic-Segmentation-1} \cite{Semantic-Segmentation-2},
трекинг объектов\cite{Pattern-Recognition} и другие.

Многие существующие методы решения задачи SOD используют только информацию, полученную из RGB изображения.
Но, кроме неё, возможно использовать и информацию с карты глубины(Depth Map) - изображения, которое хранит расстояние от запечетлённых объектов до камеры.
Карты глубины получают либо с помощью специальных камер глубины, таких как ToF-камер (Time of Flight) или структурированных световых камер, либо с помощью
стереопары изображений. Имея карту глубины, можно получить много дополнительной информации, которая будет полезна для решения задачи SOD. Ведь естественно предположить,
что объекты на переднем плане должны быть более заметны, чем те, что находятся дальше от наблюдателя.

В данной работе мы исследуем эволюцию подходов для решения задачи RGBD-SOD: от традиционных подходов до свёрточных нейронных сетей, обозначим самые актуальные модели, 
а также попробуем улучшить существующую State-of-the-Art модель - BBS-Net\cite{BBS}.

Структура работы следующая: первая часть работы - вступление. Во второй части будут рассмотрены основные подходы, которые применялись для решения задачи RGBD-SOD. Поскольку подавляющее большинство 
современных подходов предполагает использование свёрточных нейронных сетей, третья часть посвящена описанию их основных элементов, а также рассмотрению
архитектур, использующихся в качестве экстракторов признаков в других моделях. Четвёртая часть работы в деталях описывает архитектуру модели BBS-Net\cite{BBS} - одной
из State-of-the-Art модели, построенной на основе свёрточных нейронных сетей. Именно эту модель мы и попытаемся улучшить. Пятая глава
описывает предлагаемые улучшения, проведённые эксперименты и их результаты, 
а также данные, на которых эксперименты проводились. В заключительной части
подводится итог работы. 