\section{Вступление}

Задача выявления заметных объектов на изображении (Salient object detection, SOD problem) направлена на обнаружение и выделение частей изображения, наиболее привлекающих внимание
человека \cite{Is-Depth-Really-Necessary-for-SOD}. Можно сказать, что решая её, мы пытаемся имитировать визуальное восприятие человека для определения одного 
или нескольких объектов, привлекающих наибольшее внимание. Она играет ключевую роль в широком спектре задач компьютреного зрения, таких как:
распознавание изображений (Image Understanding) \cite{Image-Understanding}, стереоконструкция (Stereo Matching)\cite{Stereo-Matching},
распознавание действий(Action Recognition) \cite{Action-Recognition}, семантическая сегментация(Semantic Segmentation) \cite{Semantic-Segmentation-1} \cite{Semantic-Segmentation-2},
трекинг объектов(Object Tracking)\cite{Pattern-Recognition} и другие.

Многие существующие методы решения задачи SOD используют только информацию, доступную на RGB изображении.
Но, кроме неё, возможно использовать и информацию с карты глубины(depth map) - изображения, которое хранит расстояние от запечетлённых объектов до камеры.
Карты глубины получают либо с помощью специальных камер глубины, таких как ToF-камеры (Time of Flight) или структурированные световые камеры [TODO ссылки], либо с помощью
стереопары изображений. Имея карту глубины можно получить много дополнительной информации, которая будет полезна для решения задачи SOD, ведь естественно предположить, что объекты на переднем плане должны быть более заметны, 
чем те, что находятся дальше от наблюдателя.

Задача SOD, для решения которой кроме RGB изображения используют карту глубины, называют RDB-D saliency detection problem.
Из-за распространениея устройств, способных строить карты глубины - в основном современных смартфонов - задача выявления заметных объектов на изображении на основе RGB-изображения и карты глубины (RGBD-SOD) привлекает всё большее внимание исследователей. [TODO]

[TODO: кратко описать общие подходы и рассказать про проблемы?]

В данной работе мы рассмотрим существующие подходы для решения задачи RGB-D SOD ... и попробуем улучшить одну из State-of-the-art моделей.

Структура работы следующая:  ...