\section{Что-то про нейронки}

Наверное надо подходить с того, что мы будем использовать 

Мол сначала в ргбд есть такие-то подходы и такие модели. Мы рассматрвиаем одну из - ббс нет. Это нейронка, которая состоит из блаблабла

И вот теперь можно поподробней про свёртку, функцию активации и про части самой сетки.

\subsection{Про бекбоны}

\subsection{Функции потерь}

Начать с того, что часто в статьях+ссылки использовался BCE.

Так как задача является задачей сегментации немножка можно попробовать лосс функции, характерные для них, такие как

...

В общем что их объединяет (что типа смотрят на перекрытие и всё такое)

Ниже описываем каждую функцию отдельно: формула, описание. Мб даже график и пример работы какой-нибудь? 

\begin{itemize}
    \item BCE
    \item dice
    \item jaccard
    \item focal
\end{itemize}