\section{Что-то про нейронки}

Сегодня свёрточные нейронные сети (CNN) - г

Современные подходы к решению задачи СОД какой-то глагол/активно юзают модели основанные на 
свёрточных нейронках. В этой главе? дадим обзор чё такое  свёрточные сети, опишем операции?. 

Расскажем про архитектуры, которые в данной работе мы будем юзать в качестве фича экстракторов.



Типичной конфигурацией свёрточной сети 

.. операция свёртки, после которой применяется функция активации.

про свёртку..

про релу..

Блоки из свёрток и активаций дополняются операцией пулинга?  блин чё как сложнааааа

Главным кирпичок является - свёртка / операция свёртки.
// вот тут видмо её надо определеить и пример мб привести

// Дальше про виды свёрток. Мол кроме традиционной есть ещё такая и такая  

// Рука об руку со свёрткой иде пулинг. Определение. Бывает пулинг не простой а глобальный

// Про батч норму , релу и резидьал? 


// TODO

а может 
CNN       -- как фича экткстратктор и основа
 - операции
 - экстракторы
    - RESNET
    - Effnet
BBS
А в экспериментах где-нибудь написать про лоссы и метрики.

//

\subsection{Свёрточные Нейронные Сети}

Я хер его знает с чего начать

Свёрточные сети стали SOta подходом при работе с изображениями / при решении различных задачах, связанных с изображениями.

Так получилось потому что каскад свёрток оч хорошо извлекает разные признаки и понимает картиночку. ?? 




Общее про свёртки.

Про всякие деконвы тож можно
// TODO побольше про dilaion rate: https://towardsdatascience.com/types-of-convolutions-in-deep-learning-717013397f4d


//TODO: про апсэмплинг

//TODO: про релу? 

// операцию глобального пулинга

Потом скажем, что в нужной нам модели встраиваются готовые предобученные модели. И дальше рассказываем про них
\subsubsection{Архитектура ResNet}
\subsubsection{Архитектура Efficient Net}  --
    -- можно бекбоны прям так сильно и на описывать. Просто в двух предложениях сказать какая разница.

