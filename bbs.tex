\section{BBS}


Летом 2020 года вышла статья \cite{BBS} предлагающая новую архитектуру нейронной сети для решения задачи выявления заметных объектов на изображении 
с использованием карты глубины (RGB-D SOD). Архитектура предполагала использование сразу двух предобученных свёрточных сетей
для выделения признаков (backbones), которые применяются и к RGB, и к карте глубины. Сеть, извлекающая признаки
из карты глубины, "делится" со второй сетью картами признаков, полученных на разных этапах работы через специальный DEM модуль, 
предложенный в статье. 

Свёрточные сети, извлекающие признаки, разделены на 5 подблоков, эмулируя(?) многоуровневую систему выделения признаков с исходных изображений.
На каждом уровне блок сети, извлекающей признаки из карты глубины, передаёт получившуюся карту признаков блоку соседней сети, 
находящемуся на том же уровне (ссылка на картинку). Передача осуществляется через дополнительный модуль DEM, предложенный в работе.
Таким образом получается, что на каждом уровне признаки, полученные из RGB изображения "обогощаются" признаками из карты глубины.

После этого, карты признаков поступает в два каскадных декодера (cascaded decoder) согласно специальной стратеги. В первый декодер ($F_{CD1}$)
поступают карты признаков с 3,4 и 5-го уровней, а во второй  ($F_{CD2}$) - с 1,2 и 3-го. При этом, до поступления во второй декодер, к каждой карте низкоуровневых 
признаков (1-3 уровня) добавляется ещё и первичная карта заметности (saliency map),которая становится результатом работы первого декодера (смотри картинку)

Механизм использования первичной карты заметности для уточнения и улучшения финальной карты называется механизмом каскадного уточнения (Cascaded Refinement Mechanism).
Мотивация для его использования лежит в наблюдении за особенностями признаков, получаемых на рызных уровнях: высокоуровненвые признаки содержат семантическую информацию,
которая помогает локализовать искомый выделяющийся объект. Тогда как низкоуровневые признаки предоставляют информацию для построения более точных границ на карте значимости.







//TODO: а чего там дальше происходит?

Что-то сложное и непонятное 




 // TODO: сослаться на backbone-independent. и сказать , что хотим попробовать более современный и лёгкий бэкбон