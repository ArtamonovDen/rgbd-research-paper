\section{BBS}

Подойти в sod-approach к теме ббс. Здесь описать чё за моделька, из чего состоит

Летом 2020 года вышла статья \cite{BBS} предлагающая новую архитектуру нейронной сети для решения задачи выявления заметных объектов на изображении 
с использованием карты глубины (RGB-D SOD). Архитектура предполагала использование сразу двух предобученных свёрточных сетей
для выделения признаков (backbones), которые применяются и к RGB, и к карте глубины. Сеть, извлекающая признаки
из карты глубины, "делится" со второй сетью картами признаков, полученных на разных этапах работы через специальный DEM модуль, 
предложенный в статье.  //TODO: а чего там дальше происходит?

Что-то сложное и непонятное 




 // TODO: сослаться на backbone-independent. и сказать , что хотим попробовать более современный и лёгкий бэкбон