\section{BBS}


Летом 2020 года вышла статья \cite{BBS} предлагающая новую архитектуру нейронной сети для решения задачи выявления заметных объектов на изображении 
с использованием карты глубины (RGB-D SOD). Архитектура предполагала использование сразу двух предобученных свёрточных сетей
для выделения признаков (backbones), которые применяются и к RGB, и к карте глубины. Сеть, извлекающая признаки
из карты глубины, "делится" со второй сетью картами признаков, полученных на разных этапах работы через специальный DEM модуль, 
предложенный в статье. 


// TODO: добавить нотации с названиями уровней как функций как в статье

Свёрточные сети, извлекающие признаки, разделены на 5 подблоков, эмулируя(?) многоуровневую систему выделения признаков с исходных изображений.
На каждом уровне блок сети, извлекающей признаки из карты глубины, передаёт получившуюся карту признаков $f_i^d, i \in \{1,2,..,5\}$ блоку соседней сети, 
находящемуся на том же уровне (ссылка на картинку), который строит карту $f_i^{rgb}, i \in \{1,2,..,5\}$ . Передача осуществляется через дополнительный модуль DEM, предложенный в работе,
который применяется только к признакам из карты глубины $f_i^d, i \in \{1,2,..,5\}$. 
После этого соответствующие карты признаков поэлементно складываются, образуя кросс-модальные(cross-modality) карты признаков для каждого уровня:

\begin{equation}
    f_i^{cm} = f_i^{rgb} \bigoplus DEM(f_i^{d}), i \in \{1,2,..,5\}
\end{equation}

Таким образом получается, что на каждом уровне признаки, полученные из RGB изображения "обогощаются" признаками из карты глубины.


После этого, карты признаков поступает в два каскадных декодера (cascaded decoder) согласно специальной стратеги. В первый декодер ($F_{CD1}$)
поступают карты признаков с 3,4 и 5-го уровней, а во второй  ($F_{CD2}$) - с 1,2 и 3-го. При этом, до поступления во второй декодер, к каждой карте низкоуровневых 
признаков (1-3 уровня) добавляется ещё и первичная карта заметности (saliency map) $S_1$,которая становится результатом работы первого декодера (смотри картинку):

\begin{equation}
    S_1 = T_1(F_{CD1}(f_1^{cm},f_2^{cm},f_3^{cm}))
\end{equation}

где $F_{CD1}$ - блок первого декодера, а $T_1$ - простой модуль, содержащий 2 свёрточных слоя, который
нужен для того, чтобы уменьшить число каналов до одного и получить бинарное изображение.

Добавление первичной карты заметности (saliency map) $S_1$ к картам низкоуровневых признаков представляет собой
 поэлементное умножение:

\begin{equation}
    f_i^{cm'} = f_i^{cm} \bigotimes S_1, i \in \{1,2,3\}
\end{equation}
Таким образом, мы словно накладываем маску и оставляем из всей карты признаков только ту часть, которая соответствует 
сформированной карте заметности.

Механизм использования первичной карты заметности для уточнения и улучшения финальной карты называется механизмом каскадного уточнения (Cascaded Refinement Mechanism).
Мотивация для его использования лежит в наблюдении за особенностями признаков, получаемых на рызных уровнях: высокоуровненвые признаки содержат семантическую информацию,
которая помогает локализовать искомый выделяющийся объект. Тогда как низкоуровневые признаки предоставляют информацию для построения более точных границ на карте значимости.

Далее, карты обновлённые карты признаков $f_i^{cm'}, i \in \{1,2,3\} $ поступает во второй блок декодера $F_{CD2}$, который также оборачивается дополнительными PTM модулем,
состоящим из комбинации свёрток $1 \times 1$, нормализации по покету (batch norm), функции активации и деконволюции (смотри картинку):

\begin{equation}
    S_2 = T_2(F_{CD2}(f_1^{cm'},f_2^{cm'},f_3^{cm'}))
\end{equation}


Рассмотрим подробнее архитектуру каскадного декодера и модуля DEM

\subsection{Каскадный декодер}
Основная задача декодера - получить искомую карту заметности из карт признаков разного уровня, полученных слиянием признаков, извлечённых из карты глубины и обработанных DEM модулем, и признаков
полученных из RGB изображения.
Therefore, we introduce a light-weight cascaded decoder \url{https://arxiv.org/abs/1904.08739}  - не особо понятная штука. - декодер из другой статьи? можно это и опустить Наверное

Каскадный декодер содержит три модуля глобального контекста (Global Context Module, GCM) - по одному на каждую входную карту, после применения которых
используется метод агрегации всех карт признаков в одну. 

Модуль GCM является усовершенствованной версией модуля рецептивного поля(Receptive Fields Block, RFB), предложенного в работе \cite{RFB} (https://arxiv.org/abs/1711.07767)
Усовершенствание: 
Specifically, it contains an additional branch to enlarge the
receptive field and a residual connection [68] to preserve the
informatio

или лучше просто опистаь сначала? а про

Блок GCM состоит из четырёх параллельных веток. Для каждой из них сначала применяется свёртка $1 \time 1$, чтобы уменьшить размер 
канала до 32. После этого для 3 из 4-х веток $k, k \in \{2,3,4\}$ применяется свёртка с размером ядра(kernel size) $2k-1$. За ней 
следует расширенная свёртка(dilated convolutions) $3 \time 3$ с параметром скорости расширения (dilation rate) равным $2k-1$.
Далее, выходы всех чётырёх ветвей конкатенируются вместе, и к ним применяется свёртка $1 \time 1$. 
Результат конкатенируется с исходной картой признаком, полученной из остаточной связи исходной и финальной картами. (residual connection). // TODO подредачить. жалко чё-т картинки нет
Финальную карту признаков, полученную в результате работы GCM обозначим $f_i^{gcm'}$:

\begin{equation}
    f_i^{gcm} = F_{GCM}(f_i), i \in \{1,2,3\}
\end{equation}

Далее, к полученным картам признаков применяется специальная стратегия группировки и объединения с целью получить единственную
карту заметности. Стратегия включает в себя  поуровневое перемножение и конкатенацию разных признаков, начиная с верхнеуровневых признаков
и заканчивая низкоуровневыми. Стратегия соединения карт признаков представлена на рисунке [TODO ссылка на рисунок]

Каждая карта $f_i^{gcm'}$(кроме самой верхнеуровневой) домножается на все верхнеуровневые карты признаков:

\begin{equation}
    f_i^{gcm'} = f_i^{gcm} \bigotimes \prod_{k=i+1}^{k_{max}}Conv(F_{UP}(f_k^{gcm}))
\end{equation}

где $i \in \{1,2,3\}$, $k_{max}=3$ для первого блока декодера с картами признаков $f_i^{cm}, i \in \{1,2,3\}$ , и
$i \in \{3,4,5\}$, $k_{max}=5$ для второго блока с $f_i^{cm}, i \in \{3,4,5\}$. Операция $F_{UP}$ отображает
операцию апсэмплинга (upsampling). Операция $Conv$ - отображает свёртку $3 \time 3$.
На последнем шаге декодера к обновлённым картам признаков $f_i^{gcm'}$  применяется дополнительные операции свёртоки они конкатенируются
в одну карту признаков. 

// TODO: переделатьЕВыходная карта дополнительная пропускается через несколько BConv3 блоков и разворачивается в карту заметности:

\begin{equation}
    S = T([f_k^{gcm'}; Conv(F_{UP}[f_{k+1}^{gcm'}; Conv(F_{UP}(f_{k+2}^{gcm'}))])])
\end{equation}

где $S$ - карта заметности, $[x; y]$ - обозначает операцию конкатенации, $k \in \{1,2,3\}$. Операция $T$ - постобработка выхода декодера,
различается для первого и второго декодеров. На первом шаге применяется несколько блоков BConv3 ($T_1$). На втором шаге $T_2$ отображает 
PTM (progressively transposed module) блок для генерации финальной карты заметности с прогрессивным повышением частоты дискретизации // WTF?? - ) in a progressive upsampling way - поменять!
//TODO: Блок PTM содержит два резидьал  транспосед блоков и 3 последовательные свёртки $1 \time 1$. Каждый резидьал бэйсд блок (смотри картинку)содержит свёртку $3 \time 3$
и residual based transposed convolution ,???


// TODO -> заменить Conv на BConv, чтобы поточнее и как на картинке было?
// TODO: проверить корректность, где мы говорим только об одном, гле о двух декодерах
// TODO: блок/модуль. что одно писать? 


\subsection{Depth-Enhanced Module (DEM}

