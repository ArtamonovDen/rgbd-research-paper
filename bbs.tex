\section{BBS}


Летом 2020 года вышла статья \cite{BBS} предлагающая новую архитектуру нейронной сети для решения задачи выявления заметных объектов на изображении 
с использованием карты глубины (RGB-D SOD). Архитектура предполагала использование сразу двух предобученных свёрточных сетей
для выделения признаков (backbones), которые применяются и к RGB, и к карте глубины. Сеть, извлекающая признаки
из карты глубины, "делится" со второй сетью картами признаков, полученных на разных этапах работы через специальный DEM модуль, 
предложенный в статье. 


// TODO: добавить нотации с названиями уровней как функций как в статье

Свёрточные сети, извлекающие признаки, разделены на 5 подблоков, эмулируя(?) многоуровневую систему выделения признаков с исходных изображений.
На каждом уровне блок сети, извлекающей признаки из карты глубины, передаёт получившуюся карту признаков $f_i^d, i \in \{1,2,..,5\}$ блоку соседней сети, 
находящемуся на том же уровне (ссылка на картинку), который строит карту $f_i^{rgb}, i \in \{1,2,..,5\}$ . Передача осуществляется через дополнительный модуль DEM, предложенный в работе,
который применяется только к признакам из карты глубины $f_i^d, i \in \{1,2,..,5\}$. 
После этого соответствующие карты признаков поэлементно складываются, образуя кросс-модальные(cross-modality) карты признаков для каждого уровня:

\begin{equation}
    f_i^{cm} = f_i^{rgb} \bigoplus DEM(f_i^{d}), i \in \{1,2,..,5\}
\end{equation}

Таким образом получается, что на каждом уровне признаки, полученные из RGB изображения "обогощаются" признаками из карты глубины.


После этого, карты признаков поступает в два каскадных декодера (cascaded decoder) согласно специальной стратеги. В первый декодер ($F_{CD1}$)
поступают карты признаков с 3,4 и 5-го уровней, а во второй  ($F_{CD2}$) - с 1,2 и 3-го. При этом, до поступления во второй декодер, к каждой карте низкоуровневых 
признаков (1-3 уровня) добавляется ещё и первичная карта заметности (saliency map) $S_1$,которая становится результатом работы первого декодера (смотри картинку):

\begin{equation}
    S_1 = T_1(F_{CD1}(f_1^{cm},f_2^{cm},f_3^{cm}))
\end{equation}

где $F_{CD1}$ - блок первого декодера, а $T_1$ - простой модуль, содержащий 2 свёрточных слоя, который
нужен для того, чтобы уменьшить число каналов до одного и получить бинарное изображение.

Добавление первичной карты заметности (saliency map) $S_1$ к картам низкоуровневых признаков представляет собой
 поэлементное умножение:

\begin{equation}
    f_i^{cm'} = f_i^{cm} \bigotimes S_1, i \in \{1,2,3\}
\end{equation}
Таким образом, мы словно накладываем маску и оставляем из всей карты признаков только ту часть, которая соответствует 
сформированной карте заметности.

Механизм использования первичной карты заметности для уточнения и улучшения финальной карты называется механизмом каскадного уточнения (Cascaded Refinement Mechanism).
Мотивация для его использования лежит в наблюдении за особенностями признаков, получаемых на рызных уровнях: высокоуровненвые признаки содержат семантическую информацию,
которая помогает локализовать искомый выделяющийся объект. Тогда как низкоуровневые признаки предоставляют информацию для построения более точных границ на карте значимости.

Далее, карты обновлённые карты признаков $f_i^{cm'}, i \in \{1,2,3\} $ поступает во второй блок декодера $F_{CD2}$, который также оборачивается дополнительными PTM модулем,
состоящим из комбинации свёрток $1 \times 1$, нормализации по покету (batch norm), функции активации и деконволюции (смотри картинку):

\begin{equation}
    S_2 = T_2(F_{CD2}(f_1^{cm'},f_2^{cm'},f_3^{cm'}))
\end{equation}


Рассмотрим подробнее архитектуру каскадного декодера и модуля DEM