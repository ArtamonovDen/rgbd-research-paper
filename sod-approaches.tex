\section{Эволюция подходов к решению задачи SOD}

//TODO встпление адекватное

Решая задачу SOD исследователи дошли до использования карты глубины как способу получить пространственную информацию об объектах на изображении.

За это время были предложены различные подходы к решению этой задачи: от использования [традиционный модеои, гаусс блабла] до использования глубоких свёрточных сетей[несколько ссылок на всякие сетки] и атоэнкодеров[ссылка на UNet]

Разделяют традиционные и глубокие модели.


Традиционные алгоритмы, не использующие нейронные сети, основаны на извлечении и анализе заранее определенных признаков RGB изображения и соответствующей ему карты глубины.
Например, такими признаками может быть контрастность, цвет или границы определённой части изображения. 
В работе \cite{Depth-really-Matters} была предложена модель, использующая геометрические признаки поверностей объектов в помещении и глубину. 
[TODO: можно что-то ещё, но правда непонятно, тчо они делали]
А в работе \cite{Depth-View-of-Saliency} для расчета заметности объекта используют цветовой контраст и вычисление нормали к поверхности\cite{Surface-Normal} для каждого пикселя на изображении.


[Можно добавить картинку из https://arxiv.org/pdf/2008.00230.pdf и краткое описание моделек]


С развитием глубоких нейронных сетей на смену традиционным подходам к решению задачи SOD пришли модели, основанные на различных архитектурах нейронных сетей.
Так, в работе \cite{RGBD-SOD-Deep-Fusion} был предложена модель DF на основе свёрточной нейронной сети (CNN). Однако модель применялась не к сырому изображению, 
а к предварительно подготовленным признакам. ИЗ извлечённых признаков формировались каналы изображения, которые и передавались в нейронную сеть, которая, в свою очередь,
и выявляла самый заметный регион на картинке.

// TODO можно картинку с  \cite{RGBD-SOD-Deep-Fusion}


Последующие предложенные модели использовали более современный end-to-end подход к использованию CNN, предоставляя нейронной сети самой извлекать 
и высокоуровневые, и низкоуровневые признаки с изображений самостоятельно и без предварительных ручных шагов.
В отличие от заранее заданных признаков, признаки, извлекаемые нейронной сетью, содержат больше семантической информации об объектах на изображении, а значит
позволяют точнее решать задачи, для которых необходимое "понимать" изображённую сцену. 

Первой моделью, использующей только свёрточные сети для извлечения признаков, стала модель CTMF \cite{CNNs-Based}. 
Следующие работы представляли эксперименты с различными архитектурами сетей и различными подходами 
к совместной обработке извлечённых признаков из RGB изображении и карты глубины. В \cite{Progressively} для обработки 
изображения и карты глубины авторы применяют две различные сети, а затем сливают 2 карты признаков в одну и 
на её основе модель делает предсказание. А в работе \cite{Single-Stream} авторы предлагают противоположный предедущему 
single-stream подход к обработке изображения и его карты глубины с помощью архитектуры Single Stream Recurrent Convolution Neural Network(SSRCNN),
принимающей на вход 4 канала: 3 канала отвечают за RGB изображение, 4-й - за карту глубины.

Некоторые работы предлагают способы борьбы с зашумлённой картой глубины, которая может существенно повлиять на качество модели.
Например, в \cite{Contrast} был предложен метод по увеличению контрастности карты глубины. А в \cite{Rethinking-RGBD} был предложен 
способ автоматической оценки качества карты глубины и отфильтровывания карт плохого качества.

Кроме моделей, архитектуры которых основаны на свёрточных нейронных сетях, для задачи RGB-D SOD предлагаются и другие решения. 
Например, модель UC-Net, предложенная в работе \cite{UC-Net}, представляет собой вероятностную модель, основанную надо
 вариационный автокодировщик(VAE), чтобы эмулировать неопределённость выбора человека во время выделения самого заметного объекта.
Для каждого входа модель генерирует сразу несколько карт значимости (saliency maps) путём сэмплирования из скрытого пространства,
сформированного при обучении модели.


В данной работе мы сосредоточимся на модели BBS-Net\cite{BBS}, предложенной летом 2020 года и являющейся одним из State-of-the-Art решений.
Модель использует предобученные свёрточные сети для извлечения признаков и использует специальные модули для объединения карт признаков,
полученных из RGB изображения и карты глубины. 


//TODO: в следующей части рассмотрим общие штуки про нейронки и свёрточные сети, операции и бекбоны. А потом рассмотрим архитектуру

В следующей части мы подробнее рассмотрим архитектуру этой модели. //И преджожим варианты
её улучшения? //