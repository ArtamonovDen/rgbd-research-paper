\section{Датасеты}

С растущим интересом к задаче RGB-SOD, а также с распространением устройств, способных 
снимать изображением с картой глубины, за последние несколько лет появилось несколько RGB-D SOD датасетов.

Каждый датасет состоит из изображения с соответствующей ему картой глубины, а также
бинарное изображение с маской искомого выделяющегося объекта на изображении.

//TODO побольше про девайсы расписать!

// TODO примерчик с картиночками

//TODO описание поподробней бы

Ниже представлен краткий обзор существующих датасетов:
\begin{itemize}
    \item STERE \cite{STERE} - первый датасет со стереоскопическими изображениями, собранных с сайтов Flickr, NVIDIA 3D Vision Live и Stereoscopic Image Gallery
    Содержит 1000 изображений с различными разрешениями // TODO: добавить ссылки-сноски
    \item GIT \cite{GIT} - содержит 80 картинок домашней обстановки, собранных роботом.
    \item DES \cite{DES} - небольшой датасет объёмом 135 изображений, собранных с помошью Microsoft Kinect. Разрешение изображений: 640х480.
    \item NLPR \cite{NLPR} - содержит 1000 изображений разрешением 640х480, также собранных с помошью Microsoft Kinect.
    \item NJU2K \cite{NJU2K} - самый большой датасет, содержащий 1985 изображений с разными разрешениями, собранных из интернета, 3D фильмов и снятых на стерео камеру Fuji W3.
    \item SSD \cite{SSD} - содержит 80 картинок с высоким разрешением 960х1080, собранных из 3D фильмов.
    \item DUT-RGBD \cite{DUT} - содержит 1200 изображений со сложными для модели факторами, например: однотипный передний план или похожие передние и задние планы
    \item SIP \cite{Rethinking-RGBD} - содержит 1000 изображений с разрешением 992х744, снятых на смартфон
\end{itemize}

В таблице [ссылка на таблицу] представлены характеристики датасетов, год выхода, количество изображений и их разрешение.

// TODO таблчика как в https://arxiv.org/pdf/2008.00230.pdf или в https://arxiv.org/pdf/1907.06781.pdf